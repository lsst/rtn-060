\documentclass[OPS,authoryear,toc]{lsstdoc}
\input{meta}

% Package imports go here.

% Local commands go here.

%If you want glossaries
%\input{aglossary.tex}
%\makeglossaries

\title{Supporting Computational Science with Rubin LSST}

% Optional subtitle
% \setDocSubtitle{A subtitle}

\author{%
Knut Olsen\\
\and
Sierra Villarreal\\
}

\setDocRef{RTN-060}
\setDocUpstreamLocation{\url{https://github.com/lsst/rtn-060}}

\date{\vcsDate}

% Optional: name of the document's curator
% \setDocCurator{The Curator of this Document}

\setDocAbstract{%
More than 100 people from institutions around the world participated in a virtual workshop, "Supporting Computational Science with Rubin LSST," on March 21-22, 2023. The workshop welcomed anyone involved in the process of turning LSST data into science, especially those requiring specific computational hardware or software. The meeting focused on science use cases with Rubin LSST, and how these might be paired with specific In-Kind contributed computing resources (Independent Data Access Centers and Scientific Processing Centers (IDACs and SPCs). The workshop was followed by a two-day technical discussion between members of the Rubin IDACs Coordination Group and Rubin Data Management, aimed at understanding how to meet the demand from the community use cases and identifying technical challenges.  In this report, we summarize the outcome of the scientific workshop and the technical discussion, and identify some of the next steps needed for continuing progress.
}

% Change history defined here.
% Order: oldest first.
% Fields: VERSION, DATE, DESCRIPTION, OWNER NAME.
% See LPM-51 for version number policy.
\setDocChangeRecord{%
  \addtohist{1}{YYYY-MM-DD}{Unreleased.}{Knut Olsen}
}


\begin{document}

% Create the title page.
\maketitle
% Frequently for a technote we do not want a title page  uncomment this to remove the title page and changelog.
% use \mkshorttitle to remove the extra pages

% ADD CONTENT HERE
% You can also use the \input command to include several content files.
\section{Introduction}
The Rubin Legacy Survey of Space and Time (LSST) has partnered with In-Kind contributed resources in order to enable science above and beyond what can be provided by the US Data Facility. This has taken the form of the Independent Data Access Centers and Scientific Processing Centers (IDACs and SPCs). To begin the process of connecting Rubin data product users and their science use cases to the IDACs and SPCs, we held the ``Supporting Computational Science with Rubin LSST'' workshop on 21-22 March, 2023, followed by a two-day technical discussion within the Rubin IDACs Coordination Group and Rubin Data Management to understand the challenges raised by this workshop.

The workshop focused primarily on science use cases within the Rubin community and identifying the requirements that these represented. This includes which data products are necessary, the size of data products, requisite software requirements, analysis expectations, compute and storage needs, and implementation plans. We ultimately retrieved 48 science use cases in areas including the solar system, Milky Way, nearby galaxies, galactic and extragalactic transients and variables, galaxies and Active Galactic Nuclei (AGN), lensing, and cosmology.

These use cases provided information for the technical group follow-up these needs and how they lined up with Rubin Operations expectations. This included discussion on user support, data distribution mechanisms, data storage requirements, lite catalogue definitions, and underlying service requirements. In particular, the need to define a lite catalogue was recognized as a high priority task, alongside with making certain IDACs are prepared to receive data. The discussion here provided several avenues of rich follow-up data.

\section{Science Use Case Themes}
Here we talk about the science use cases that were provided to us and the themes that were recognized.

\section{User Expectations and Realities}
Here we talk about user xpectations and the realities of how the IDACs are currently situated.

\section{Lite Catalogue}
Here we talk about the lite catalogue and how it may or may not meet science collaboration needs.

\section{Hardware and Software Requirements}
Here we talk about hardware and software requirements, with a particular eye for data transfer.

\section{Conclusions}
Here we reiterate the main points from above and put together a few concrete bullet point next steps gained out of this.

\appendix
% Include all the relevant bib files.
% https://lsst-texmf.lsst.io/lsstdoc.html#bibliographies
\section{References} \label{sec:bib}
\renewcommand{\refname}{} % Suppress default Bibliography section
\bibliography{local,lsst,lsst-dm,refs_ads,refs,books}

% Make sure lsst-texmf/bin/generateAcronyms.py is in your path
\section{Acronyms} \label{sec:acronyms}
\input{acronyms.tex}
% If you want glossary uncomment below -- comment out the two lines above
%\printglossaries





\end{document}
